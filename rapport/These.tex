\documentclass[a4paper,11pt,twoside]{StyleThese}

\usepackage{amsmath,amssymb}             % AMS Math
\usepackage[french]{babel}
\usepackage[latin1]{inputenc}
\usepackage[T1]{fontenc}
\usepackage[left=1.5in,right=1.3in,top=1.1in,bottom=1.1in,includefoot,includehead,headheight=13.6pt]{geometry}
\renewcommand{\baselinestretch}{1.05}

% Table of contents for each chapter

\usepackage[nottoc, notlof, notlot]{tocbibind}
\usepackage[french]{minitoc}
\setcounter{minitocdepth}{2}
\mtcindent=15pt
% Use \minitoc where to put a table of contents

\usepackage{aecompl}

% Glossary / list of abbreviations

\usepackage[intoc]{nomencl}
\renewcommand{\nomname}{Liste des Abr�viations}

\makenomenclature

% My pdf code

\usepackage{ifpdf}

\ifpdf
  \usepackage[pdftex]{graphicx}
  \DeclareGraphicsExtensions{.jpg}
  \usepackage[a4paper,pagebackref,hyperindex=true]{hyperref}
\else
  \usepackage{graphicx}
  \DeclareGraphicsExtensions{.ps,.eps}
  \usepackage[a4paper,dvipdfm,pagebackref,hyperindex=true]{hyperref}
\fi

\graphicspath{{.}{images/}}

%nicer backref links
\renewcommand*{\backref}[1]{}
\renewcommand*{\backrefalt}[4]{%
\ifcase #1 %
(Non cit�.)%
\or
(Cit� en page~#2.)%
\else
(Cit� en pages~#2.)%
\fi}
\renewcommand*{\backrefsep}{, }
\renewcommand*{\backreftwosep}{ et~}
\renewcommand*{\backreflastsep}{ et~}

% Links in pdf
\usepackage{color}
\definecolor{linkcol}{rgb}{0,0,0.4} 
\definecolor{citecol}{rgb}{0.5,0,0} 

% Change this to change the informations included in the pdf file

\hypersetup
{
bookmarksopen=true,
pdftitle="Cr�ation et utilisation d'atlas anatomiques num�riques pour la radioth�rapie",
pdfauthor="Olivier COMMOWICK", %auteur du document
pdfsubject="Segmentation d'images par atlas et cr�ation d'atlas", %sujet du document
%pdftoolbar=false, %barre d'outils non visible
pdfmenubar=true, %barre de menu visible
pdfhighlight=/O, %effet d'un clic sur un lien hypertexte
colorlinks=true, %couleurs sur les liens hypertextes
pdfpagemode=None, %aucun mode de page
pdfpagelayout=SinglePage, %ouverture en simple page
pdffitwindow=true, %pages ouvertes entierement dans toute la fenetre
linkcolor=linkcol, %couleur des liens hypertextes internes
citecolor=citecol, %couleur des liens pour les citations
urlcolor=linkcol %couleur des liens pour les url
}

% definitions.
% -------------------

\setcounter{secnumdepth}{3}
\setcounter{tocdepth}{2}

% Some useful commands and shortcut for maths:  partial derivative and stuff

\newcommand{\pd}[2]{\frac{\partial #1}{\partial #2}}
\def\abs{\operatorname{abs}}
\def\argmax{\operatornamewithlimits{arg\,max}}
\def\argmin{\operatornamewithlimits{arg\,min}}
\def\diag{\operatorname{Diag}}
\newcommand{\eqRef}[1]{(\ref{#1})}

\usepackage{rotating}                    % Sideways of figures & tables
%\usepackage{bibunits}
%\usepackage[sectionbib]{chapterbib}          % Cross-reference package (Natural BiB)
%\usepackage{natbib}                  % Put References at the end of each chapter
                                         % Do not put 'sectionbib' option here.
                                         % Sectionbib option in 'natbib' will do.
\usepackage{fancyhdr}                    % Fancy Header and Footer

% \usepackage{txfonts}                     % Public Times New Roman text & math font
  
%%% Fancy Header %%%%%%%%%%%%%%%%%%%%%%%%%%%%%%%%%%%%%%%%%%%%%%%%%%%%%%%%%%%%%%%%%%
% Fancy Header Style Options

\pagestyle{fancy}                       % Sets fancy header and footer
\fancyfoot{}                            % Delete current footer settings

%\renewcommand{\chaptermark}[1]{         % Lower Case Chapter marker style
%  \markboth{\chaptername\ \thechapter.\ #1}}{}} %

%\renewcommand{\sectionmark}[1]{         % Lower case Section marker style
%  \markright{\thesection.\ #1}}         %

\fancyhead[LE,RO]{\bfseries\thepage}    % Page number (boldface) in left on even
% pages and right on odd pages
\fancyhead[RE]{\bfseries\nouppercase{\leftmark}}      % Chapter in the right on even pages
\fancyhead[LO]{\bfseries\nouppercase{\rightmark}}     % Section in the left on odd pages

\let\headruleORIG\headrule
\renewcommand{\headrule}{\color{black} \headruleORIG}
\renewcommand{\headrulewidth}{1.0pt}
\usepackage{colortbl}
\arrayrulecolor{black}

\fancypagestyle{plain}{
  \fancyhead{}
  \fancyfoot{}
  \renewcommand{\headrulewidth}{0pt}
}

\usepackage{MyAlgorithm}
\usepackage[noend]{MyAlgorithmic}

%%% Clear Header %%%%%%%%%%%%%%%%%%%%%%%%%%%%%%%%%%%%%%%%%%%%%%%%%%%%%%%%%%%%%%%%%%
% Clear Header Style on the Last Empty Odd pages
\makeatletter

\def\cleardoublepage{\clearpage\if@twoside \ifodd\c@page\else%
  \hbox{}%
  \thispagestyle{empty}%              % Empty header styles
  \newpage%
  \if@twocolumn\hbox{}\newpage\fi\fi\fi}

\makeatother
 
%%%%%%%%%%%%%%%%%%%%%%%%%%%%%%%%%%%%%%%%%%%%%%%%%%%%%%%%%%%%%%%%%%%%%%%%%%%%%%% 
% Prints your review date and 'Draft Version' (From Josullvn, CS, CMU)
\newcommand{\reviewtimetoday}[2]{\special{!userdict begin
    /bop-hook{gsave 20 710 translate 45 rotate 0.8 setgray
      /Times-Roman findfont 12 scalefont setfont 0 0   moveto (#1) show
      0 -12 moveto (#2) show grestore}def end}}
% You can turn on or off this option.
% \reviewtimetoday{\today}{Draft Version}
%%%%%%%%%%%%%%%%%%%%%%%%%%%%%%%%%%%%%%%%%%%%%%%%%%%%%%%%%%%%%%%%%%%%%%%%%%%%%%% 

\newenvironment{maxime}[1]
{
\vspace*{0cm}
\hfill
\begin{minipage}{0.5\textwidth}%
%\rule[0.5ex]{\textwidth}{0.1mm}\\%
\hrulefill $\:$ {\bf #1}\\
%\vspace*{-0.25cm}
\it 
}%
{%

\hrulefill
\vspace*{0.5cm}%
\end{minipage}
}

\let\minitocORIG\minitoc
\renewcommand{\minitoc}{\minitocORIG \vspace{1.5em}}

\usepackage{multirow}

\newenvironment{bulletList}%
{ \begin{list}%
	{$\bullet$}%
	{\setlength{\labelwidth}{25pt}%
	 \setlength{\leftmargin}{30pt}%
	 \setlength{\itemsep}{\parsep}}}%
{ \end{list} }

\newtheorem{definition}{D�finition}
\renewcommand{\epsilon}{\varepsilon}

% centered page environment

\newenvironment{vcenterpage}
{\newpage\vspace*{\fill}\thispagestyle{empty}\renewcommand{\headrulewidth}{0pt}}
{\vspace*{\fill}}



\begin{document}

\begin{titlepage}
\begin{center}
\noindent {\large \textbf{UNIVERSIT� DE NICE - SOPHIA ANTIPOLIS}} \\
\vspace*{0.3cm}
\noindent {\LARGE \textbf{�COLE DOCTORALE STIC}} \\
\noindent \textbf{SCIENCES ET TECHNOLOGIES DE L'INFORMATION \\ ET DE LA COMMUNICATION} \\
\vspace*{0.5cm}
\noindent \Huge \textbf{T H � S E} \\
\vspace*{0.3cm}
\noindent \large {pour obtenir le titre de} \\
\vspace*{0.3cm}
\noindent \LARGE \textbf{Docteur en Sciences} \\
\vspace*{0.3cm}
\noindent \Large de l'Universit� de Nice - Sophia Antipolis \\
\noindent \Large \textbf{Mention : \textsc{Informatique}}\\
\vspace*{0.4cm}
\noindent \large {Pr�sent�e et soutenue par\\}
\noindent \LARGE Olivier \textsc{Commowick} \\
\vspace*{0.8cm}
\noindent {\Huge \textbf{Cr�ation et utilisation d'atlas anatomiques num�riques pour la radioth�rapie}} \\
\vspace*{0.8cm}
\noindent \Large Th�se dirig�e par Gr�goire \textsc{Malandain} \\
\vspace*{0.2cm}
\noindent \Large pr�par�e � l'INRIA Sophia Antipolis, Projet \textsc{Asclepios} \\
\vspace*{0.2cm}
\noindent \large soutenue le 14 f�vrier 2007 \\
\vspace*{0.5cm}
\end{center}
\noindent \large \textbf{Jury :} \\
\begin{center}
\noindent \large 
\begin{tabular}{llcl}
      \textit{Rapporteurs :}	& Patrick \textsc{Clarysse}		& - & CNRS (CREATIS)\\
				& Louis \textsc{Collins}		& - & McGill University\\
      \textit{Directeur :}	& Gr�goire \textsc{Malandain}		& - & INRIA (Asclepios)\\
      \textit{Pr�sident :}	& Nicholas \textsc{Ayache}		& - & INRIA (Asclepios)\\
      \textit{Examinateurs :}   & Pierre-Yves \textsc{Bondiau}          & - & Centre Antoine Lacassagne (Nice)\\
      				& Guido \textsc{Gerig}			& - & University of North Carolina\\
      				& Vincent \textsc{Gr�goire}		& - & Universit� Catholique de Louvain\\
      \textit{Invit� :}		& Hanna \textsc{Kafrouni}		& - & DOSISoft S.A.
\end{tabular}
\end{center}
\end{titlepage}
\sloppy

\titlepage


\dominitoc
\pagenumbering{roman}
\cleardoublepage

\section*{Remerciements}

A faire en dernier :-)

\tableofcontents

\mainmatter

\chapter{Introduction}
\label{chap:intro}
\minitoc

\nomenclature{DTI}{Diffusion Tensor Imaging}

 \cite{Commowick_MICCAI_2007}
 
 \section{Le calcul scientifique et la simulation numérique}
Un des principaux objectifs de la communauté du calcul haute performance est de produire des applications de simulation numeérique efficientes. En raison des importants besoins en puissance de calcul que requiert une simulation de phénomènes physiques.

\appendix

\chapter{Exemple d'annexe}
\label{chap:annexe1}

\section{Exemple d'annexe}

Et j'en profite pour me citer et montrer mon style bibtex (deux auteurs) \cite{Commowick_MICCAI_2007}.

Et deux autres citations pour le style optionnel de bibliographie : un seul auteur \cite{Oakes_RStat_1999} et plus de deux auteurs \cite{Guimond_CVIU_2000}.

\bibliographystyle{StyleThese}
\bibliography{These}

\printnomenclature

\cleardoublepage
\begin{vcenterpage}
\noindent\rule[2pt]{\textwidth}{0.5pt}
\\
{\large\textbf{Résumé :}}

ds


{\large\textbf{Mots clés :}}
Segmentation par atlas, recalage non linéaire, radiothérapie, création d'atlas
\\
\noindent\rule[2pt]{\textwidth}{0.5pt}
\end{vcenterpage}

\begin{vcenterpage}
\noindent\rule[2pt]{\textwidth}{0.5pt}
\begin{center}
{\large\textbf{Design and Use of Numerical Anatomical Atlases for Radiotherapy\\}}
\end{center}
{\large\textbf{Abstract:}}
The main objective of this thesis is to provide radio-oncology specialists with automatic  
\\
{\large\textbf{Keywords:}}
Atlas-based Segmentation, non rigid registration, radiotherapy, atlas creation
\\
\noindent\rule[2pt]{\textwidth}{0.5pt}
\end{vcenterpage}

\end{document}
