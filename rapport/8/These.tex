\documentclass[a4paper,11pt,twoside]{StyleThese}

\include{formatAndDefs}

\begin{document}

\begin{titlepage}
\begin{center}
\noindent {\large \textbf{UNIVERSITÉ DE NICE - SOPHIA ANTIPOLIS}} \\
\vspace*{0.3cm}
\noindent {\LARGE \textbf{ÉCOLE DOCTORALE STIC}} \\
\noindent \textbf{SCIENCES ET TECHNOLOGIES DE L'INFORMATION \\ ET DE LA COMMUNICATION} \\
\vspace*{0.5cm}
\noindent \Huge \textbf{T H È S E} \\
\vspace*{0.3cm}
\noindent \large {pour obtenir le titre de} \\
\vspace*{0.3cm}
\noindent \LARGE \textbf{Docteur en Sciences} \\
\vspace*{0.3cm}
\noindent \Large de l'Université de Nice - Sophia Antipolis \\
\noindent \Large \textbf{Mention : \textsc{Informatique}}\\
\vspace*{0.4cm}
\noindent \large {Présentée et soutenue par\\}
\noindent \LARGE Olivier \textsc{Commowick} \\
\vspace*{0.8cm}
\noindent {\Huge \textbf{Création et utilisation d'atlas anatomiques numériques pour la radiothérapie}} \\
\vspace*{0.8cm}
\noindent \Large Thèse dirigée par Grégoire \textsc{Malandain} \\
\vspace*{0.2cm}
\noindent \Large préparée à l'INRIA Sophia Antipolis, Projet \textsc{Asclepios} \\
\vspace*{0.2cm}
\noindent \large soutenue le 14 février 2007 \\
\vspace*{0.5cm}
\end{center}
\noindent \large \textbf{Jury :} \\
\begin{center}
\noindent \large 
\begin{tabular}{llcl}
      \textit{Rapporteurs :}	& Patrick \textsc{Clarysse}		& - & CNRS (CREATIS)\\
				& Louis \textsc{Collins}		& - & McGill University\\
      \textit{Directeur :}	& Grégoire \textsc{Malandain}		& - & INRIA (Asclepios)\\
      \textit{Président :}	& Nicholas \textsc{Ayache}		& - & INRIA (Asclepios)\\
      \textit{Examinateurs :}   & Pierre-Yves \textsc{Bondiau}          & - & Centre Antoine Lacassagne (Nice)\\
      				& Guido \textsc{Gerig}			& - & University of North Carolina\\
      				& Vincent \textsc{Grégoire}		& - & Université Catholique de Louvain\\
      \textit{Invité :}		& Hanna \textsc{Kafrouni}		& - & DOSISoft S.A.
\end{tabular}
\end{center}
\end{titlepage}
\sloppy

\titlepage


\dominitoc

\pagenumbering{roman}

 \cleardoublepage

\section*{Remerciements}

A faire en dernier :-)

\tableofcontents

\mainmatter

\chapter{Introduction}
\label{chap:intro}
\minitoc

\nomenclature{DTI}{Diffusion Tensor Imaging}

 
 
 \section{Le domaine du Calcul Haute Performance} 
 

%---------------------------------------------------------------------------------

 \subsection{Le calcul scientifique et la simulation numérique}

Pour comprendre d'où émerge le domaine du Calcul Haute Performance (HPC) il faut comprendre pour répondre à quels besoins ces architectures sont mises au point. 
Le domaine du calcul scientifique et notamment celui de la simulation numérique qui nécessite de granges puissances de calculs. Les simulations sont utilisées dans différents domaines car elles apportent beaucoup d'avantages. Le premier est la réduction des couts. Par exemple dans l'industrie automobile, les tests de crash de voiture ne sont plus réalisé avec de vrais voitures. Les voitures sont maintenant simulées et envoyées percuter des murs virtuels. Cette techniques à pour effet de réduire les temps de conception, car il n'y plus besoin de créer une voiture avec les matériaux à tester, et donc de réduire les couts de conceptions. Mais la simulation numérique à d'autres avantages, comme celui de pouvoir simuler des phénomènes dont les conditions ne sont pas reproductibles sur terre. Elle élargi donc les domaines explorable ce qui rend son champs d'application presque infini. Les domaines d'applications sont donc nombreux, on retrouve la simulation numérique dans la recherche pétrolière (analyse des fonds marins), les prévisions météoroligiques, en biologie (séquençage ADN) ou encore en finance.

\begin{figure}[H]
    \center
    \includegraphics[width=4cm]{images/Chapitre1/maillage.png}
    \caption{\label{maillage} Le maillage le plus fin exploité par Météo-France pour ses prévisions régionales restitue des mailles de 2,5 km de côté. (source www.irma-grenoble.com)}
\end{figure}


Ces simulations numériques utilisent une réprésentation discrète des objets modélisés. Pour améliorer ces simulations, ces réprésentation doivent utiliser des maillages le plus fin possible. C'est en cela que ces simulations necessitent d'énormes puissances de calculs et que cette demande est casi illimité, car les maillages pourront toujours etre affinés. 


%---------------------------------------------------------------------------------

\subsection{Définition du Calcul Haute Performance}
\nomenclature{HPC}{Calcul Haute Performance}
 
Le domaine du Calcul Haute Performance est l'interconnexion de ressources informatiques dans le but de résoudre de façon partagée un problème complexe. Les probleèmes qui sont résolus grâce à ces immenses systemes sont très variés et interviennent dans de nombreux domaines. Le point commun de ces application est la résolution d'un gros problème qui ne peut pas être résolu par une seule ressource. Ce problème est divisé en sous-problème de petites tailles, qui eux peuvent être résolus séparément. Cette mêthode de résolution est appelée le calcul pralallele (voir ??). 
Aujourd'hui il existe trois façon d'apporter cette puissance de calcul aux utilisateurs:
\begin{enumerate}
\item \textbf{Dedicated supercomputer}, unique est créee ce qui rend son prix très cher. Ces architectures ne sont utilisées que très rarement aujourd'hui pour des cas très précis.
\item \textbf{Commodity cluster}, qui agrège du matériel grand public pour former des grappes de calculs de plusieurs milliers de processeurs.
\item HPC dans le nuage, TODO avec Poru les nuls
\end{enumerate}

Ce regroupements de centaines, voire de milliers de ressources forme une grappe de serveur que l'on appelle un \textit{cluster} ou \textit{supercalculateur}.


\subsection{Les Clusters}
A l'origine les premiers supercalculateurs étaient des architectures uniques crées de toutes pièces pour un client. Il était alors très dure de les reproduire ensuite rendant leur cout de conception trés élevé. Seymour Cray présenta le premier super- calculateur en 1960 alors qu'il travaillait pour Control Data Corporation. A partir des années 1990 apparurent des clusters construits à partir de materiels, certes haut de gamme, mais qui constituent les ordinateurs grand public. C'est seulement le regroupement de centaines de stations de travail qui en fait des supercalculateur, et cette façon de les construire est toujours la même aujourda'hui.

\cite{Ste95}


\subsection{Loi de Moore A DEPLACER}
En 1965, Gordon Moore fit l'une des prédictions les plus visionnaires de toute l'histoire de l'informatique (Moore, 1965) lorsqu'il énonça que la performance des ordinateurs doublerait tous les dix-huit mois. De nos jours, cette remarquable prédiction reste toujours aussi pertinente. Cependant, alors que cette amélioration des performances a longtemps permis de conserver un modele de programmation sequentiel, ces dernieres années ont vu apparaitre des sarchitectures parallèles au sein même des microprocesseurs (multicoeurs). Ce changement de conception radical au niveau du matériel, oblige à revoir les méthodes de développement logiciel afin de tirer pleinement parti de la puissance de ces nouveaux processeurs.



\appendix

\chapter{Exemple d'annexe}
\label{chap:annexe1}

\section{Exemple d'annexe}

%Et j'en profite pour me citer et montrer mon style bibtex (deux auteurs) \cite{Commowick_MICCAI_2007}.
hihi
%Et deux autres citations pour le style optionnel de bibliographie : un seul auteur \cite{Oakes_RStat_1999} et plus de deux auteurs \cite{Guimond_CVIU_2000}.

\bibliographystyle{StyleThese}
\bibliography{These}

%\printnomenclature

\cleardoublepage
\begin{vcenterpage}
\noindent\rule[2pt]{\textwidth}{0.5pt}
\\
{\large\textbf{Résumé :}}
L'objectif de cette thèse est de fournir aux radiothérapeutes des outils de contourage automatique des structures à risque pour la planification de la radiothérapie des tumeurs cérébrales et de la région ORL.
\\
Nous utilisons pour cela un atlas anatomique, constitué d'une représentation de l'anatomie associée à une image de celle-ci. Le recalage de cet atlas permet de contourer automatiquement les organes du patient et ainsi obtenir un gain de temps considérable. Les contributions présentées se concentrent sur trois axes.
\\
Tout d'abord, nous souhaitons obtenir une méthode de recalage la plus indépendante possible du réglage de ses paramètres. Celui-ci, effectué par le médecin, se doit d'être minimal, tout en garantissant un résultat robuste. Nous proposons donc des méthodes de recalage permettant un meilleur contrôle de la transformation obtenue, en passant par des techniques de rejet d'appariements aberrants ou en utilisant des transformations localement affines.
\\
Le second axe est consacré à la prise en compte de structures dues à la tumeur. En effet, la présence de ces structures, absentes de l'atlas, perturbe le recalage de celui-ci. Nous proposons donc également des méthodes afin de contourer ces structures et de les prendre en compte dans le recalage.
\\
Enfin, nous présentons la construction d'un atlas ORL et son évaluation sur une base de patients. Nous montrons ici la faisabilité de l'utilisation d'un atlas de cette région, ainsi qu'une méthode simple afin d'évaluer les méthodes de recalage utilisées pour construire un atlas.
\\
L'ensemble de ces travaux a été implémenté dans le logiciel Imago de DOSIsoft, ceci ayant permis d'effectuer une validation en conditions cliniques.
\\
\\
{\large\textbf{Mots clés :}}
Segmentation par atlas, recalage non linéaire, radiothérapie, création d'atlas
\\
\noindent\rule[2pt]{\textwidth}{0.5pt}
\end{vcenterpage}

\begin{vcenterpage}
\noindent\rule[2pt]{\textwidth}{0.5pt}
\begin{center}
{\large\textbf{Design and Use of Numerical Anatomical Atlases for Radiotherapy\\}}
\end{center}
{\large\textbf{Abstract:}}
The main objective of this thesis is to provide radio-oncology specialists with automatic tools for delineating organs at risk of a patient undergoing a radiotherapy treatment of cerebral or head and neck tumors.
\\
To achieve this goal, we use an anatomical atlas, i.e. a representative anatomy associated to a clinical image representing it. The registration of this atlas allows to segment automatically the patient structures and to accelerate this process. Contributions in this method are presented on three axes.
\\
First, we want to obtain a registration method which is as independent as possible w.r.t. the setting of its parameters. This setting, done by the clinician, indeed needs to be minimal while guaranteeing a robust result. We therefore propose registration methods allowing to better control the obtained transformation, using outlier rejection techniques or locally affine transformations.
\\
The second axis is dedicated to the consideration of structures associated with the presence of the tumor. These structures, not present in the atlas, indeed lead to local errors in the atlas-based segmentation. We therefore propose methods to delineate these structures and take them into account in the registration.
\\
Finally, we present the construction of an anatomical atlas of the head and neck region and its evaluation on a database of patients. We show in this part the feasibility of the use of an atlas for this region, as well as a simple method to evaluate the registration methods used to build an atlas.
\\
All this research work has been implemented in a commercial software (Imago from DOSIsoft), allowing us to validate our results in clinical conditions.
\\
\\
{\large\textbf{Keywords:}}
Atlas-based Segmentation, non rigid registration, radiotherapy, atlas creation
\\
\noindent\rule[2pt]{\textwidth}{0.5pt}
\end{vcenterpage}

\end{document}
